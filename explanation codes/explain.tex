\documentclass{article}
\usepackage[utf8]{inputenc}
\usepackage{enumitem}
\usepackage{karnaugh-map}
\usepackage{tikz}
\usepackage{karnaugh-map}
\usepackage{graphicx}
\usepackage{subfig}
\usepackage{circuitikz}
\documentclass{article}
\usetikzlibrary{arrows,shapes.gates.logic.US,shapes.gates.logic.IEC,calc}


\title{ASSIGNMENT   6(2)}
\author{lmadhuripagolu }
\date{December 2020}

\begin{document}

\maketitle

\section{QUESTION}
The boolean expression F(X,Y,Z)=$\overline{X}Y\overline{Z}+X\overline{Y}\overline{Z}+XY\overline{Z}+XYZ$ converted into the canonical product of sum(POS) form is
\section{ANSWER}
\subsection{table}
\begin{table}[ht]
\begin{tabular}{|c|c|c|c|c|}
\hline
X & Y & Z & F & maxterms\\

0 & 0 & 0 & 0 & X+Y+Z \\

0 & 0 & 1 & 0 & X+Y+$\overline{Z}$\\

0 & 1 & 0 & 1 & -\\

0 & 1 & 1 & 0 & $X+\overline{Y}+\overline{Z}$ \\

1 & 0 & 0 & 1 & -\\

1 & 0 & 1 & 0 & $\overline{X}+Y+\overline{Z}$\\

1 & 1 & 0 & 1 & -\\

1 & 1 & 1 & 1 & -\\
\hline
\end{tabular}
    \caption{truth table}
    \label{tab:my_label}
\end{table}


product of sum =(X+Y+Z)(X+Y+$\overline{Z}$)(X+$\overline{Y}+\overline{Z})$)($\overline{X}+Y+\overline{Z})$(as \ product \ of \ sum \ is \ the\ combination \ for \ F=0 \\\\
From the table we can see that when (X,Y,Z)=0, ((X,Y)=0,Z=1),(X=0,(Y,Z)=1) and ((X,Z=1,Y=0) we get output i.e F=0.\\\\\
So the max terms are F(X,Y,Z)=M_0.M_1.M_3.M_5(suffixes\ denote \ decimal \ code)\\
where M_1=(X+Y+Z) ; M_2=(X+Y+$\overline{Z}$) ; M_3=(X+$\overline{Y}+\overline{Z})$) ; M_3+($\overline{X}+Y+\overline{Z})$\\\\
Min terms,F(X+Y+Z)=M_2+M_4+M_6+M_7\\
where M_2=$\overline{X}Y\overline{Z} ; M_4=X\overline{Y}\overline{Z} ; M_6=XY\overline{Z} ; M_7=XYZ$









\subsection{k-map}
\begin{karnaugh-map}[4][2][1][YZ][X]
        \minterms{2,4,7,6}
        \maxterms{0,1,3,5}
        \indeterminants{2,5}
        \implicant{0}{1}
        \implicant{1}{3}
        \implicant{1}{5}
        
 \end{karnaugh-map}   
from the k-map Product of Sum(POS) for F=$(Y+X)(\overline{Z}+Y)(\overline{Z}+X)$\\
The min terms are given 1 and max terms are given 0.As product of sum is given by max terms we need to group zeros in the kmaap\\
\\\ There is only possibility of doubles in the respective k-map.So we need to map (1,5) i.e vertically , (0,1) and (1,3) horizontally.\\\

\subsection{logic gates} 

\tikzstyle{branch}=[fill,shape=circle,minimum size=3pt,inner sep=0pt]
\begin{tikzpicture}[label distance=5mm]
\node (Z) at (0,0) {Z};
\node (Y) at (1,0) {Y};
\node (X) at (2,0) {X};
    \node[not gate US, draw, rotate=-90] at ($(Z)+(0.5,-1)$) (Not2) {};
    \node[not gate US, draw, rotate=-90] at ($(Y)+(0.5,-1)$) (Not1) {};
    \node[not gate US, draw, rotate=-90] at ($(X)+(0.5,-1)$) (Not0) {};
    \node[or gate US, draw, logic gate inputs=nnn] at ($(x0)+(4,-2)$) (Or1) {};
    \node[or gate US, draw, logic gate inputs=nnn] at ($(Or1)+(0,-1)$) (Or2) {};
    \node[or gate US, draw, logic gate inputs=nnn] at ($(Or2)+(0,-1)$) (Or3) {};
    \node[or gate US, draw, logic gate inputs=nnn] at ($(Or3)+(0,-1)$) (Or4) {};
    \node[and gate US, draw, logic gate inputs=nnnn, anchor=input 1] at ($(Or3.output)+(3,1)$) (And1) {};
     \foreach \i in {2,1,0}
    {
        \path (x\i) -- coordinate (punt\i) (x\i |- Not\i.input);
        \draw (punt\i) node[branch] {} -| (Not\i.input);}
        \draw (Z) |- (Or1.input 1);
        \draw (Not2.output) |- (Or2.input 2);
        \draw (Not2.output) |- (Or3.input 2);
        \draw (Not2.output) |- (Or4.input 2);
        \draw (Y) |- (Or1.input 2);
        \draw (Y) |- (Or2.input 1);
        \draw (Y) |- (Or4.input 1);
        \draw (Not1.output) |- (Or3.input 1);
        \draw (X) |- (Or1.input 3);
         \draw (X) |- (Or2.input 3);
          \draw (X) |- (Or3.input 3);
          \draw (Not0.output) |- (Or4.input 3);
          \draw (Or1.output) -- (And1.input 1);
          \draw (Or2.output) -- (And1.input 2);
          \draw (Or3.output) -- (And1.input 3);
          \draw (Or4.output) -- (And1.input 4);
          \draw (And1.output) -- ([xshift=0.5cm]And1.output);
         
         
          
          
          
     
          


        
         
        
    
    
    \end{tikzpicture}
    \end{document}
    



